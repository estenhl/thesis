\documentclass[thesis.tex]{subfiles}


\begin{document}
\chapter*{Abstract}
\textcolor{red}{A reference genome is a database containing a representative sample of the genetic information of a species. As sequencing technologies improve we able to produce a larger amount of digitalized genetic data. To account for the innate variation in this data, new models for representing the information is being proposed. We will in this thesis discuss graphs as a model for reference genomes.}\\
\par\noindent
We will present a new generalized method for aligning text strings against a graph based reference genome. The method is based on the concept of context-based mapping, \textcolor{red}{a technique propsed to standardize uniqueness in structures which does not have an innate coordinate system.} The method has been made accessible through a tool which is available online.\\
\par\noindent
We tested the feasibility of the approach by doing performance comparisons with existing methods. The results show several traits of the approach which outperforms \textcolor{red}{THE OTHERS}. We argue that the method provides a viable solution to \textcolor{red}{the most general version of the problem,} which can provide a basis for more specific biological applications.\\
\begin{itemize}
  \item Motivation behind graphs?
  \item More detail on reference genomes?
  \item More technical terms?
  \item Mention validation?
\end{itemize}
\end{document}