\documentclass[thesis.tex]{subfiles}


\begin{document}
\chapter*{Abstract}
As sequencing technologies improve, we are able to produce a larger amount of genetic data. One of the models used to organize and store this data are reference genomes, structures which collect such information to form a representative sample of the genome for a given species. To account for the variation which appears as the amount of data increases, new models for representing reference genomes are necessary. Graphs present the opportunity to have complex interrelationships between elements, a property which naturally solves the problem of variation. The newest human reference genome, GRCh38, already incorporate graph-like features through the introduction of alternate paths through variable regions. Methods created for interacting with the existing structures are traditionally centered around linear data representations, realized as a set of text string operations. To allow a complete transition, these methods must be adapted to fit the domain of graphs. \\
\par\noindent
An important string operation in the context of genetic data is sequence alignment. In reference genomes, this is a technique which can be utilized for mapping new data against the reference. In this thesis, we present a new method for aligning text strings against graph based reference genomes. The method is built on the concept of context-based mapping; a technique proposed to standardize uniqueness in structures which do not have an innate coordinate system. We have made the method accessible through a tool which is available online.\\
\par\noindent
We test the feasibility of our approach by doing performance comparisons with existing methods, examining both accuracy and efficiency. The results display several traits of the approach which outperform other proposed solutions. We argue that the method provides a viable solution to the most general version of the problem, which provides a basis for more specific biological applications.\\
\end{document}