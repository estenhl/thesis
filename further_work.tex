\documentclass[thesis.tex]{subfiles}

\begin{document}
\chapter{Future work}
\label{sec:future_work}
\textcolor{red}{In this chapter we will present what we assume to be necessary work to utilize we have developed as a stepping stone for finding a better solution to the alignment problem when the reference genome is modelled as a graph.} We will do this at various levels of abstraction, starting out with the explicit changes needed in the implementation to create a tool viable for answering actual biological questions. There are several details which would need to be introduced, such as the possibility of searching for reverse complementary strings and knowing the origin of the input data. We consider these domain-specific, and trivial to implement. Then we face the problem of our index quickly growing out of control. This is a more intricate problem to solve, but a necessary one to make the approach feasible for real-life datasets. A correlated problem emerges from the exponential growth in time complexity. This is a problem which can be tackled on several levels of technical detail: The most general solution would be to build a more efficient index. Another, more conceptually tractable, solution could be to incoorporate restrictions on the input data to avoid the most complex searches. The heuristical algorithm we propose has the inherent poperty of doing the latter whenever it is faced with problem instances which cannot be solved non-heuristically. Utilizing the knowledge that some results are heuristical could for instance be of importance in an assembly process.\\
\par\noindent
Our heuristical algorithm strictly kicks in whenever the error margin is too low. \textcolor{red}{This is a due to a ``flaw'' in our approach, not necessarily general to the problem itself.} One could however imagine situations where the data is simply so large and complex it is impossible to search through it all. We envision this theoretical case such that no amount of optimization would make a full search tractable. In this case we propose that instead of improving the search one builds in a self-validation of the system, a way of scoring the heuristically found alignments. This can be a far less computationally complex operation based on existing scoring routines for regular strings, for instance based around scoring the separate entities making up the alignment. We imagine any such information could be of great importance for an assembly process.\\
\par\noindent
Due to the generality of the problem we defined, the testing done in this thesis has been done on strictly generic data. It would be interesting to, preferably after implementing the improvements mentioned in the first section, test the approach on real genetic data. Determining the value of the flexibility of graph based approaches should be done by comparing it to the regular, linear existing approaches. We boldly claim this would present another dimension of both precision and analytical opportunities.
\end{document}