\documentclass[thesis.tex]{subfiles}

\begin{document}
\chapter{Future work}
\label{sec:future_work}
In this chapter we will present our thoughts around what we assume to be necessary adjustments and expansions to the approach presented in this thesis to solve specific biological problems concerned with alignment problems where the reference genome is a graph. We will do this at various levels of abstraction, starting out with the explicit changes needed in the implementation to create a tool viable for answering actual biological questions. There are several details which would need to be introduced, such as the possibility of searching for reverse complementary strings and storing the origin of the input data. We consider these domain-specific, and trivial to implement. At this level of abstraction we also include the optimization to parallelization which is sketched out in section \ref{sec:impl_parallelization}.\\
\par\noindent
The rapid growth seen in the size of the index is a more conceptual problem to solve, but one we consider necessary to make the approach feasible for real life datasets. We displayed some possible solutions in section \ref{sec:discussion_indexation}. A related problem is the one concerned with the exponential growth in time complexity with regards to the fuzzyness. This is a property which lies in the index, as it reflects the exponential increase in possibilities directly on the number of computations. Because the size of the index is already problematic, the naive solution of increasing its complexity is not satisfying. One possible solution to this problem is to decrease the number of searches. This can for instance be done by searching for larger structures, much like Paten et al. when they look up entire contexts instead of singular vertices. Our algorithm works as it does in order to preserve a level of flexibility until the exhaustive search. We believe this is a problem which can be solved by devising an algorithm for finding combinations of larger sequences, which will need to be sophisticated but can still see present an overall decrease in complexity.\\ 
\par\noindent
We see the heuristical modification of the algorithm as the most interesting concept to continue pursuing. In section \ref{sec:heuristical_applications} we briefly sketch out systems for scoring the heuristical alignments. An actual implementation of this concept can go a long way to avoid having to deal with the complex problem instances which require high error margins. When these can be avoided, the algorithm as we present it depicts a powerful solution.\\
\par\noindent
Another interesting property which emerges from the heuristical version of the alignment is the possibility for aligning split sequences. We especially consider this interesting in the cases where the alignments are not split apart by a large gap, but when the split happens across distinct branches in the graph. These are the cases when a naive distance metric is no longer intricate enough, which is when the flexibility presented by graphs can really unfold.
\end{document}