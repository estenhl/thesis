\documentclass[thesis.tex]{subfiles}

\chapter{Aims}
The project in itself had a clear goal: Develop an algorithm for aligning against graph based reference genomes. This thesis will not be concerned with the chronological events of the development process. Instead the thesis will be concerned with presenting the result of the project: The algorithm ``Fuzzy context-based search''. Interesting design choices taken throughout the process will be presented through the algorithm itself, the reasoning behind these choices given as formal arguments underway. Additionally, the thesis has two smaller goals:
\begin{itemize}
  \item Validate the correctness of the approach
  \item Perform performance testing and comparisons to other tools on larger datasets
\end{itemize}
In order to succeed with the two smaller goals, a tool was created which implements the algorithm. This tool is available online, instructions on retrieving and using the tool can be found in Appendix \ref{sec:tool}.\\
\par\noindent
Throughtout the development process we were faced with several decisions regarding the specificity of the problem. In many of these situations we chose to put an upper bound to the complexity, to end up with a simple, general, formally strict proof-of-concept, which should work as a base for later expansions into more specific applications. Some of these might seem as ``shortcuts'' to the reader: We assure this is not the case. Every time the result of one of these simplifications is presented we defend it. In the later parts of the thesis we reintroduce many of the when discussing the feasibility of the approach in relation to more specific biological problems.\\
\par\noindent
During the master project the article ``Canonical, Stable, General Mapping using Context Schemes'' was published by Novak et al. \cite{canonical_stable_general_mapping_using_context_schemes}, discussing an approach to alignment which is similar to the one presented in this thesis. The similarities and differences between the two is granted a large part of the discussion section of the thesis.