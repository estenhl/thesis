\documentclass[thesis.tex]{subfiles}

\begin{document}
\chapter{Introduction}
With the initial sequencing of the human genome in 2001 \cite{human_genome}, the blueprint of our species was digitalized and made publicly available. Researches utilize this information to understand the cause of genetic diseases and disorders, and have made great progress in developing of more accurate diagnoses and treatment. In order to understand the data, bioinformaticians have developed tools and methods for analysis. These techniques are typically string algorithms developed to handle traditional linear genetic sequences. As sequencing technology progresses, the cost of sequencing sees a proportionate decrease\textcolor{red}{[ref]}. This will in turn lead to a larger number of individuals being sequenced to form a larger base of genetic information. Because DNA is prone to mutations, the sequencing data is bound to contain variation. The amount of variation present in a database is directly impacted by the number of sequenced individuals and will therefore grow at a rate correlated to the progression in the sequencing technologies.\\
\par\noindent
In order to account for the variation, more complex standards for modeling genetic data is being proposed. One possibility for such a model revolves around graph based representations\textcolor{red}{[ref]}, a structure far superior to text strings regarding flexibility. The value of this approach might seem indisputable, but this is based on an assumption that it can do everything the old model can, and then some. Deciding the feasibility of existing operations adapted to fit the new model is crucial in order to determine its validity.\\
\par\noindent
The sequence alignment problem revolves around finding similarities between strings and is situated at the core of genetic analysis. The problem has for some time been considered solved when the input data is regular strings\textcolor{red}{[ref]}. We will in this thesis present our translation of the alignment problem to the realm of graphs and present an approach for solving this variant of the problem.
\end{document}