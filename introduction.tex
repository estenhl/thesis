\documentclass[thesis.tex]{subfiles}

\begin{document}
\chapter{Introduction}
This thesis is the final result of a master project in Informatics: Programming and networks. The project has been supervised by the Biomedical Informatics research group at the Department of Informatics, University of Oslo and was a part of the early phases of a larger project regarding graph-based reference genomes.
\section{Motivation}
We started out with an open exploration of the possibilities of using graph representations for reference genomes. Throughout the early phases the need for a formalization of the alignment process became apparent. Even something as basic as deciding what the graphs would look like could be solved through defining this process, by implicitly deciding how the input data should be compared and combined. The existing straight-forward approach was not satisfying, and alignment stood out as the most pressing problem to solve to get the ball rolling. 
\section{Aims of the thesis}
The project in itself had a clear goal: Develop an algorithm for alignment against graph-based reference genomes. This thesis will not be concerned with the chronological events of the development process. Instead the thesis will be concerned with presenting the result of the project: The algorithm ``Fuzzy context-based search''. Interesting design choices taken throughout the process will be presented through the algorithm itself, the reasoning behind the decisions motivated by formal arguments given underway. Additionally, the thesis has two smaller goals:
\begin{itemize}
  \item Validate the correctness of the approach
  \item Perform performance testing and comparisons to other tools on larger datasets
\end{itemize}
In order to succeed with the two smaller goals, we implemented the algorithm in the \textit{GraphGenome} tool. This tool is available online, instructions on retrieving and using it can be found in Appendix \ref{sec:tool}.\\
\par\noindent
Throughtout the development process we were faced with several decisions regarding the specificity of the problem. In many of these situations we chose to put an upper bound to the complexity, to end up with a simple, general, formally strict proof-of-concept, which should work as a base for later expansions into more specific applications. Some of these might seem as ``shortcuts'' to the reader: We assure this is not the case. Every time the result of one of these simplifications is presented we defend it. In the later parts of the thesis we reintroduce many of the when discussing the feasibility of the approach in relation to more specific biological problems.\\
\par\noindent
During the master project the article ``Canonical, Stable, General Mapping using Context Schemes'' was published, discussing an approach to alignment which is similar to the one presented in this thesis. The similarities and differences between the two is granted a large part of the discussion section of the thesis.
\section{Contents}
The first chapter will present the theoretical background for the thesis: Why is there a problem which needs to be solved, what kind of data are we dealing with and more specific explorations of the previous progress on the subject. The third and fourth chapters are concerned with presenting the algorithm, first as a conceptual overview and then in more technical detail through explaining the implementation found in the tool. The next chapter addresses the validation of the algorithm. This is done through a series of examples run on the tool and can thus also be used as an introductionary manual. Chapter 6 tests the performance of the tool by running a large number of tests on large, real-life datasets. The results from both these chapters are discussed in Chapter 7, along with the previously mentioned comparison of two approaches to the problem. In the eight chapter a conclusion is drawn regarding the value of the approach before finally possible future improvements is discussed in Chapter 9.
\end{document}