\documentclass[thesis.tex]{subfiles}

\begin{document}
\chapter{Introduction}
Ever since the initial sequencing of the human genome in 2001 \cite{human_genome}, the blueprint to OUR SPECIES have been digitalized and publicly available. Researches utilize this information for SOMETHING IMPORTANT. In order to understand the data, bioinformaticians have developed tools and methods for ANALYZING IT. These techniques are typically string algorithms made to handle traditional linear genetic sequences. As sequencing technology progresses, the cost of sequencing sees a proportionate decrease. This will in turn lead to a larger number of individuals being sequenced to form a larger base of genetic information. Because the SOMETHING is prone to mutations, the data is bound to contain variation. The amount of variation present is directly impacted by the number of individuals and will therefore grow at a rate correlated to the progression in the sequencing technologies. 
\end{document}