\documentclass[thesis.tex]{subfiles}

\begin{document}
\chapter{Introduction}
This thesis is the final result of a master project in Informatics: Programming and networks. The project has been supervised by the Biomedical Informatics research group at the Department of Informatics, University of Oslo and was a part of the start phase of a larger project regarding graph-based reference genomes.
\section{Motivation}
The project started out as an exploration of graph representations of reference genome. Throughout the early phases the need for an alignment algorithm presented itself heavily. Even something as basic as deciding what the graphs should look like would be solved through defining this process, by implicitly deciding how the data should be combined. Alignment stood out as the most pressing problem to solve to get the ball rolling, and the existing straight-forward approach was not satisfying. 
\section{Aims of the thesis}
The project in itself had a clear goal: Develop an algorithm for aligning against graph-based reference genomes. This thesis will not be concerned with the chronological events of the development process. Instead the thesis will be concerned with presenting the result of the project: The algorithm ``Fuzzy context-based search''. Interesting design choices taken throughout the process will be presented through the algorithm itself, the reasoning behind these choices given as formal arguments underway. Additionally, the thesis has two smaller goals:
\begin{itemize}
  \item Validate the correctness of the approach
  \item Perform performance testing and comparisons to other tools on larger datasets
\end{itemize}
In order to succeed with the two smaller goals, a tool was created which implements the algorithm. This tool is available online, instructions on retrieving and using the tool can be found in section \ref{sec:tool} of the appendix.
\par\noindent
During the master project the article ``Canonical, Stable, General Mapping using Context Schemes'' was published, discussing an approach to alignment which is similar to the one presented in this thesis. The similarities and differences between the two is granted a large part of the discussion section of the thesis.
\section{Contents}
The first chapter will present the theoretical background for the thesis: What kind of data are we dealing with, why is there a problem which needs to be solved and what is done in the field previously. The third and fourth chapters are concerned with presenting the algorithm, first as a conceptual overview and then in more technical detail through explaining the implementation found in the tool. The next chapter addresses the validation of the algorithm. This is done through a series of examples run in the tool and can thus also be used as an introductionary manual. Chapter 6 tests the performance of the tool by running a large number of tests on large, real-life datasets. The results from both these chapters are discussed in Chapter 7, along with the previously mentioned comparison of two approaches to the problem. In the eight chapter a conclusion is drawn regarding the value of the approach before finally possible future improvements is discussed in Chapter 9.
\end{document}