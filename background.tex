\documentclass[thesis.tex]{subfiles}

\begin{document}
\chapter{Background}
\section{DNA}
\textit{Deoxyribonucleic acid} (DNA) is a molecule in which living organisms store genetic information. The information is encoded by \textit{nucleotides} bound together by a sugar-phosphate backbone into strands. The nucleotides are smaller molecules which can contain one of the nitrogenous bases \textit{\textbf{A}denine}, \textit{\textbf{C}ytosine}, \textit{\textbf{G}uanine} or \textit{\textbf{T}hymine}. Each of the bases are complementary to another base, A with T and C with G. Due to the chemical structure of the nucleotides, a DNA strand can be said to have a direction: Upstream towards the 5' end or downstream towards the 3' end. DNA strands can be connected with a \textit{reverse complementary} strand in a double helix. The two strands will have opposing directions, and every base in one of the strands will be connected to its complement in the other. Because either of the strands are easily deduced from the other, DNA is usually represented by only of them. Because DNA can be seen as a linear sequence of discrete units it can be represented by text strings containing the four leading letters of each nucleotide. The text strings representations often also contain the letter N, referencing \textit{aNy base}.
\par\noindent
The smallest unit used for representing DNA is the nucleotide. Several continuous nucleotides make up a \textit{contig}. Several contigs, not necessarily overlapping or continuous, make up a \textit{scaffold}. Several scaffolds can again be combined into \textit{chromosomes} which are the building blocks of a complete \textit{genome}. The genome is the unit containing the genetic information for a single individual.
\subsection{Variation}

\subsection{The human genome}
\subsubsection{Major Histocompatibility Complex}
\subsection{Sequencing}
\subsection{Mapping}
\subsection{Alignment}
\section{Sequence graphs}
\subsection{Representation}
\subsection{Mapping}
\subsection{Alignment}
\section{Techniques and tools}
\subsection{Suffix trees}
\subsection{Graph representations}
\subsection{Visualization of graphs}
\end{document}