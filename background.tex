\documentclass[thesis.tex]{subfiles}

\begin{document}
\chapter{Background}
\section{Genetics}
\textit{Deoxyribonucleic acid} (DNA) is a molecule in which living organisms store genetic information. The information is encoded by \textit{nucleotides} bound together by a sugar-phosphate backbone into strands. The nucleotides are smaller molecules which contain one of the nitrogenous bases \textit{\textbf{A}denine}, \textit{\textbf{C}ytosine}, \textit{\textbf{G}uanine} or \textit{\textbf{T}hymine}. Each of the bases are complementary to another, A with T and C with G. Due to the chemical structure of the nucleotides, a DNA strand can be said to have a direction: Upstream towards the 5' end or downstream towards the 3' end. DNA strands can be connected with a \textit{reverse complementary} strand in a double helix. The two strands will have opposing directions, and every base in one of the strands will be connected to its complement. The paired nucleotides are called \textit{base pairs}. Because either of the strands are easily deduced from the other, DNA is usually represented by only of them. DNA can be seen as a linear sequence of discrete units and can thus be represented by text strings, containing the four leading letters representing nucleotides. The text strings representations often also contain the letter N, referencing \textit{aNy base}.
\subsection{Gene}
\textcolor{red}{"What is a gene?" Helen Pearson}
\subsection{Variation}
Genetic information is prone to mutations, either as a result of environmental influence or as a consequence of imperfections in reproduction. The simplest mutations are \textit{point mutations} which affect a single nucleotide base. Point mutations can either be \textit{Single-nucleotide polymorphisms} (SNPs) where a single base is substituted for another, or \textit{insertions} or \textit{deletions} (indels) where a single nucleotide is removed or inserted into the genetic sequence. Mutations can also occur over larger areas of the genome, where longer subsequences can be deleted, inserted, moved or reversed. A final type of mutations is \textit{Copy number variations} where a longer sequence of DNA, typically at least 1 kb \cite{copy_number_variation_new_insights_in_genome_diversity}, is repeated a variable number of times.\\
\subsection{Reference genomes}
\subsection{The human genome}
The human genome consists of roughly 3 billion base pairs (bp). These base pairs are spread over 22 paired chromosomes and is assumed to contain about 23 000 genes \cite{introduction_to_genomics}. The current human reference genome is GRCh38, developed and maintained by the \textit{Genome Reference Consortium} \textcolor{red}{HOWTO: reference websites}. GRCh38 contains 261 alternate loci, spread over 178 out of a total of 238 regions. An average human is estimated to deviate from the reference genome in 10.000-11.000 synonymous sites and 10.000-12.000 non-synonymous sites.
\subsubsection{Major Histocompatibility Complex}
The \textit{Major Histocompatibility Complex} (MHC) is a genetic region spanning approximately 4 million base pairs (mb) \cite{immunobiology_the_immune_system_in_health_and_disease}. In humans it is located on chromosome 6 and contains about 200 genes. MHC is a region known to contain genes which affect the functionality of the immune system \cite{the_importance_of_immune_gene_variability_in_evolutionary_ecology_and_conservation}. Even more so MHC is known to be a highly variable region, containing variants that are directly associated with disease \cite{variation_analysis_and_gene_annotation_of_eight_mhc_haplotypes}.
\subsection{Sequencing}
\subsection{Alignment}
\textit{Sequence alignment} is the process of determining correspondence between text strings, in this case representing DNA, by mapping the elements from one to the elements of the other. The score of an alignment is determined by a \textit{scoring schema}, which provides scores for mapping characters against characters through a \textit{substitution matrix} and penalties for introducing \textit{gaps}. A gap refers to an element in one of the strings which has no counterpart in the other string when aligned (Fig. \ref{fig:alignments}). The scoring schemas can be based around simple match/mismatch scores, which corresponds to the mathematical \textit{Edit distance problem}, or more complex scores (Fig. \ref{fig:substitution_matrix}). These complex models typically try to model the probabilities behind the physical processes responsible for change. The computational sequence alignment problem consists of finding the highest scoring alignment for any two strings. There exists two main variants of the problem, \textit{global alignments} where to entire strings are aligned against each other and \textit{local alignments} where a string is aligned against a substring of another. The two are traditionally solved respectively by the \textit{Needleman-Wunsch} and \textit{Smith-Waterman} algorithms which both are based on \textit{dynamic programming} (Section \ref{sec:dynamic_programming}).
\begin{wrapfigure}{L}{0.3\textwidth}
		\begin{subfigure}[t]{\textwidth}
			\begin{mdframed}
				\begin{center}
					\texttt{ACGGGCCTA}\\
					\texttt{||||\space||||}\\
					\texttt{ACGGACCTA}
				\end{center}
			\end{mdframed}
			\caption{An alignment with no gaps, but one mismatch}
		\end{subfigure}
		\begin{subfigure}[b]{\textwidth}
			\begin{mdframed}
				\begin{center}
					\texttt{ACGGGCCTA}\\
					\texttt{||||\space\space|||}\\
					\texttt{ACGG---CTA}
				\end{center}
			\end{mdframed}
			\caption{An alignment with a single gap of length 2}
		\end{subfigure}
	\caption{Examples of aligned text strings}
	\label{fig:alignments}
\end{wrapfigure}
\par\noindent
If more than two sequences are aligned the result is a \textit{Multiple sequence alignment} (MSA). This is typically done on sequences which is expected to share a common ancestor to determine which traits of the individuals arised from the same origins and how the involved species have diverged over time. A final variant of the alignment problem is one involving large databases of sequences, where the algorithms does not only need to find the best alignment between two sequences, but also determine which sequence should be chosen in order to maximize the result. Both of the preceding techniques utilize heuristical methods to decrease the computational complexity.
\begin{wrapfigure}{R}{0.80\linewidth}
  \begin{mdframed}
    \begin{center}
      \begin{tabularx}{\linewidth}{ccccc}
          &\texttt{A}&\texttt{C}&\texttt{G}&\texttt{T}\\ \cline{2-5}
        \texttt{A}&\multicolumn{1}{|c|}{\texttt{91}}&\texttt{-114}&\multicolumn{1}{|c|}{\texttt{-31}}&\multicolumn{1}{c|}{\texttt{-123}}\\ \cline{2-5}
        \texttt{C}&\multicolumn{1}{|c|}{\texttt{-114}}&\texttt{100}&\multicolumn{1}{|c|}{\texttt{-125}}&\multicolumn{1}{c|}{\texttt{-31}}\\ \cline{2-5}
        \texttt{G}&\multicolumn{1}{|c|}{\texttt{-31}}&\texttt{-125}&\multicolumn{1}{|c|}{\texttt{100}}&\multicolumn{1}{c|}{\texttt{-114}}\\ \cline{2-5}
        \texttt{T}&\multicolumn{1}{|c|}{\texttt{-123}}&\texttt{-31}&\multicolumn{1}{|c|}{\texttt{-114}}&\multicolumn{1}{c|}{\texttt{91}}\\ \cline{2-5}
      \end{tabularx}
    \end{center}
  \end{mdframed}
  \caption{The HOXD70 substitution matrix}
  \label{fig:substitution_matrix}
\end{wrapfigure}
\section{Graph-based genome representations}
Representing genetic information as graphs instead of the traditional linear representations have some major advantages. Graphs are far more expressive structures compared to text strings, able to represent more complex relationships between the elements involved. Secondly, if biological questions can be rephrased to graph theoretical settings, the extensive mathematical field of graph theory can present more feasible approaches to previously hard problems. There is however a major problem: A more complex structure calls for more sophisticated variants of existing methods. Graph-based approaches have been used for some time in the assembly process, and more recently in relation to reference genomes. This section will present both of these approaches alongside some of the remaining unsolved problems. No graph theoretical foreknowledge is needed as all the involved elements will be defined before they are used, but for interested readers there are exists good sources \textcolor{red}{(FIND SOME SOURCES)}.
\begin{defn}[Graph]
  A pair $G=\{V,E\}$ where $V$ is a set of vertices and $E$ is a set of edges. $|G|$ denotes the number of vertices in $G$.
\end{defn}
\subsection{Representation}
Deciding upon the representation of the graph consists of defining the structure of the elements involved, namely the vertices and edges. As the graphs are built from genetic information the basic building blocks, the nucleotides, should obviously be represented. If the input data are more complex than single nucleotides, we must represent the relationships. Because the input data has variation, the structure needs to tolerate flexibility. There is however a risk of making the structures so flexible they present no consistency, and a flexibility/rigidness-tradeoff becomes apparent (fig. \ref{fig:flexibility_rigidness}). How the structures are defined in detailed should be determined through the operations which are desirable to perform on them.

\begin{wrapfigure}{L}{0.65\textwidth}
  \begin{mdframed}
    \begin{subfigure}[t]{\textwidth}
      \begin{mdframed}
        \begin{center}
          \begin{tikzpicture}[>=stealth',shorten >=1pt,auto,node distance=1.4cm]
            \node[state] (q0) {$start$};
            \node[state] [above right of=q0] (q1) {$A$};
            \node[state] [below right of=q0] (q2) {$C$};
            \node[state] [right of=q1] (q3) {$G$};
            \node[state] [right of=q2] (q4) {$T$};
            \node[state] [below right of=q3] (q5) {$end$};

            \path (q0) edge node {} (q1)
            edge node {} (q2)
            edge node {} (q3)
            edge node {} (q4)
            edge node {} (q5)
            (q1) edge node {} (q2)
            edge node {} (q3)
            edge node {} (q4)
            edge node {} (q5)
            (q2) edge node {} (q3)
            edge node {} (q4)
            edge node {} (q5)
            (q3) edge node {} (q4)
            edge node {} (q5)
            (q4) edge node {} (q5);
          \end{tikzpicture}
        \end{center}
      \end{mdframed}
      \caption{A graph with a path corresponding to every possible DNA string}
    \end{subfigure}
    \begin{subfigure}[t]{\textwidth}
      \begin{mdframed}
        \begin{center}
          \begin{tikzpicture}[->,>=stealth',shorten >=1pt,auto,node distance=1.25cm]
            \node[state,scale=0.7] (q0) {$start$};
            \node[state,scale=0.7] [right =0.3cm of q0] (q1) {$A$};
            \node[state,scale=0.7] [right of=q1] (q2) {$G$};
            \node[state,scale=0.7] [right of=q2] (q3) {$C$};
            \node[state,scale=0.7] [right of=q3] (q4) {$T$};
            \node[state,scale=0.7] [right of=q4] (q5) {$C$};
            \node[state,scale=0.7] [right=0.3cm of q5] (q6) {$end$};
            \node[state,scale=0.7] [above of=q1] (q7) {$A$};
            \node[state,scale=0.7] [right of=q7] (q8) {$G$};
            \node[state,scale=0.7] [right of=q8] (q9) {$G$};
            \node[state,scale=0.7] [right of=q9] (q10) {$T$};
            \node[state,scale=0.7] [right of=q10] (q11) {$C$};
            \node[state,scale=0.7] [below of=q1] (q12) {$A$};
            \node[state,scale=0.7] [right of=q12] (q13) {$G$};
            \node[state,scale=0.7] [right of=q13] (q14) {$C$};
            \node[state,scale=0.7] [right of=q14] (q15) {$T$};
            \node[state,scale=0.7] [right of=q15] (q16) {$C$};

            \path (q0) edge node {} (q1)
            edge node {} (q7)
            edge node {} (q12)
            (q1) edge node {} (q2)
            (q2) edge node {} (q3)
            (q3) edge node {} (q4)
            (q4) edge node {} (q5)
            (q5) edge node {} (q6)
            (q7) edge node {} (q8)
            (q8) edge node {} (q9)
            (q9) edge node {} (q10)
            (q10) edge node {} (q11)
            (q11) edge node {} (q6)
            (q12) edge node {} (q13)
            (q13) edge node {} (q14)
            (q14) edge node {} (q15)
            (q15) edge node {} (q16)
            (q16) edge node {} (q6);
          \end{tikzpicture}
        \end{center}
      \end{mdframed}
      \caption{A graph which is built through alignments without allowing variation}
    \end{subfigure}
  \end{mdframed}
  \caption{Two graphs with vertices representing nucleotides and edges representing sequences displaying too much flexibility (a) and (arguably) too much rigidity (b)}
  \label{fig:flexibility_rigidness_tradeoff}
\end{wrapfigure}
In the article ``An Eulerian path approach to DNA fragment assembly''\cite{an_eulerian_path_approach_to_dna_fragment_assembly}, Pevzner, Tang and Waterman proposes using \textit{de Bruijn graphs} as a solution to find the correct assembly of repeats during fragment assembly. A de Bruijn graph is a structure where vertices represent \textit{k-mers} from an alphabet and edges represent relationships between the k-mers of two vertices. Pevzner et al. lets the vertices contain strings of length $l-1$ and connects vertices with an edge wherever there exists a read of length $l$ containing the two substrings. Formulating the problem in this fashion lets the problem be formulated as a \textit{Eulerian path} problem, solvable in polynomial time, rather than the traditional ``overlap-layout-consensus'' method which is equivalent to the NP-complete problem of finding a \textit{Hamiltonian path}.
\subsection{Mapping}
\subsection{Alignment}
\section{Techniques and tools}
\subsection{Dynamic programming}
\label{sec:dynamic_programming}
\subsection{Implementing graphs}
\subsection{Suffix trees}
\subsection{Visualization of graphs}
\end{document}



















