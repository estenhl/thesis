\documentclass[thesis.tex]{subfiles}

\begin{document}
\chapter{Background}
\section{Genetics}
\textit{Deoxyribonucleic acid} (DNA) is a molecule in which living organisms store genetic information. The information is encoded by \textit{nucleotides} bound together by a sugar-phosphate backbone into strands. The nucleotides are smaller molecules which contain one of the nitrogenous bases \textit{\textbf{A}denine}, \textit{\textbf{C}ytosine}, \textit{\textbf{G}uanine} or \textit{\textbf{T}hymine}. Each of the bases are complementary to another, A with T and C with G. Due to the chemical structure of the nucleotides, a DNA strand can be said to have a direction: Upstream towards the 5' end or downstream towards the 3' end. DNA strands can be connected with a \textit{reverse complementary} strand in a double helix. The two strands will have opposing directions, and every base in one of the strands will be connected to its complement. The paired nucleotides are called \textit{base pairs}. Because either of the strands are easily deduced from the other, DNA is usually represented by only of them. DNA can be seen as a linear sequence of discrete units and can thus be represented by text strings, containing the four leading letters representing nucleotides. The text strings representations often also contain the letter N, referencing \textit{aNy base}.
\subsection{Gene}
\textcolor{red}{"What is a gene?" Helen Pearson}
\subsection{Variation}
Genetic information is prone to mutations, either as a result of environmental influence or as a consequence of imperfections in reproduction. The simplest mutations are \textit{point mutations} which affect a single nucleotide base. Point mutations can either be \textit{Single-nucleotide polymorphisms} (SNPs) where a single base is substituted for another, or \textit{insertions} or \textit{deletions} (indels) where a single nucleotide is removed or inserted into the genetic sequence. Mutations can also occur over larger areas of the genome, where longer subsequences can be deleted, inserted, moved or reversed. A final type of mutations is \textit{Copy number variations} where a longer sequence of DNA, typically at least 1 kb \cite{copy_number_variation_new_insights_in_genome_diversity}, is repeated a variable number of times.\\
\subsection{Reference genomes}
\subsection{The human genome}
The human genome consists of roughly 3 billion base pairs (bp). These base pairs are spread over 22 paired chromosomes and is assumed to contain about 23 000 genes \cite{introduction_to_genomics}. The current human reference genome is GRCh38, developed and maintained by the \textit{Genome Reference Consortium} \textcolor{red}{HOWTO: reference websites}. GRCh38 contains 261 alternate loci, spread over 178 out of a total of 238 regions. An average human is estimated to deviate from the reference genome in 10.000-11.000 synonymous sites and 10.000-12.000 non-synonymous sites.
\subsubsection{Major Histocompatibility Complex}
The \textit{Major Histocompatibility Complex} (MHC) is a genetic region spanning approximately 4 million base pairs (mb) \cite{immunobiology_the_immune_system_in_health_and_disease}. In humans it is located on chromosome 6 and contains about 200 genes. MHC is a region known to contain genes which affect the functionality of the immune system \cite{the_importance_of_immune_gene_variability_in_evolutionary_ecology_and_conservation}. Even more so MHC is known to be a highly variable region, containing variants that are directly associated with disease \cite{variation_analysis_and_gene_annotation_of_eight_mhc_haplotypes}.
\subsection{Sequencing}
\subsection{Alignment}
\textit{Sequence alignment} is the process of determining correspondence between text strings, in this case representing DNA, by mapping the elements from one to the elements of the other. The score of an alignment is determined by a \textit{scoring schema}, which provides scores for mapping characters against characters and penalties for introducing \textit{gaps}. A gap refers to an element in one of the strings which has no counterpart in the other string when aligned (See fig. \ref{fig:alignments}). The scoring schemas can be based around simple match/mismatch scores, which corresponds to the mathematical \textit{Edit distance problem}, or more complex scores which model the probabilities behind the physical processes responsible for changes. The computational sequence alignment problem consists of finding the highest scoring alignment for any two strings. The problem can be solved with dynamic programmin
\begin{wrapfigure}{L}{0.3\textwidth}
		\begin{subfigure}[t]{\textwidth}
			\begin{mdframed}
				\begin{center}
					\texttt{ACGGGCCTA}\\
					\texttt{||||\space||||}\\
					\texttt{ACGGACCTA}
				\end{center}
			\end{mdframed}
			\caption{An alignment with no gaps, but one mismatch}
		\end{subfigure}
		\begin{subfigure}[b]{\textwidth}
			\begin{mdframed}
				\begin{center}
					\texttt{ACGGGCCTA}\\
					\texttt{||||\space\space|||}\\
					\texttt{ACGG----CTA}
				\end{center}
			\end{mdframed}
			\caption{An alignment with a single gap of length 2}
		\end{subfigure}
	\caption{Examples of aligned text strings}
	\label{fig:alignments}
\end{wrapfigure}
\section{Sequence graphs}
\subsection{Representation}
\subsection{Mapping}
\subsection{Alignment}
\section{Techniques and tools}
\subsection{Dynamic programming}
\subsection{Implementing graphs}
\subsection{Suffix trees}
\subsection{Visualization of graphs}
\end{document}



















